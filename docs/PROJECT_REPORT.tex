%%
%% 12-Lead ECG Reconstruction from Reduced Lead Sets
%% DATA 5000 Final Project Report
%% Team 4: Damilola Olaiya & Mithun Mani
%%
%% Based on ACM sigconf template
%%
\documentclass[sigconf]{acmart}

%% Rights management - for class project
\setcopyright{none}
\acmConference[DATA 5000]{Data Science Capstone Project}{December 2025}{Carleton University, Ottawa, Canada}
\acmYear{2025}

%% Remove ACM-specific elements for class project
\settopmatter{printacmref=false}
\renewcommand\footnotetextcopyrightpermission[1]{}
\pagestyle{plain}

%% Additional packages
\usepackage{amsmath}
\usepackage{amsfonts}
\usepackage{algorithm}
\usepackage{algorithmic}
\usepackage{multirow}
\usepackage{booktabs}
\usepackage{graphicx}
\usepackage{xcolor}

%%
%% Title
%%
\title{12-Lead ECG Reconstruction from Reduced Lead Sets: A Hybrid Physics-Informed Deep Learning Approach}

%%
%% Authors
%%
\author{Damilola Olaiya}
\email{damilolaolaiya@cmail.carleton.ca}
\affiliation{%
  \institution{Carleton University}
  \city{Ottawa}
  \state{Ontario}
  \country{Canada}
}

\author{Mithun Mani}
\email{mithunmani@cmail.carleton.ca}
\affiliation{%
  \institution{Carleton University}
  \city{Ottawa}
  \state{Ontario}
  \country{Canada}
}

\renewcommand{\shortauthors}{Olaiya \& Mani}

%%
%% Abstract
%%
\begin{abstract}
Cardiovascular disease (CVD) is the world's leading cause of death, yet the gold-standard 12-lead electrocardiogram (ECG) remains inaccessible in many settings due to equipment complexity and the need for trained personnel. We present a hybrid physics-informed deep learning approach to reconstruct the full 12-lead ECG from only 3 measured leads (I, II, and one precordial lead). Our method combines deterministic physiological relationships---Einthoven's and Goldberger's laws---for exact limb lead derivation with a 1D U-Net neural network for chest lead reconstruction. Using the PTB-XL dataset (21,837 clinical ECGs), we employ patient-wise data splits to prevent leakage and evaluate both signal fidelity (MAE, Pearson correlation, SNR) and downstream diagnostic utility through multi-label classification. Our hybrid approach guarantees perfect reconstruction of derived limb leads while achieving high-fidelity reconstruction of chest leads, preserving clinically relevant morphological features essential for accurate cardiac diagnosis.
\end{abstract}

%%
%% CCS Concepts
%%
\begin{CCSXML}
<ccs2012>
   <concept>
       <concept_id>10010147.10010178.10010179</concept_id>
       <concept_desc>Computing methodologies~Machine learning</concept_desc>
       <concept_significance>500</concept_significance>
   </concept>
   <concept>
       <concept_id>10010147.10010178.10010179.10010182</concept_id>
       <concept_desc>Computing methodologies~Neural networks</concept_desc>
       <concept_significance>500</concept_significance>
   </concept>
   <concept>
       <concept_id>10003120.10003121.10003129</concept_id>
       <concept_desc>Human-centered computing~Ubiquitous and mobile computing</concept_desc>
       <concept_significance>300</concept_significance>
   </concept>
</ccs2012>
\end{CCSXML}

\ccsdesc[500]{Computing methodologies~Machine learning}
\ccsdesc[500]{Computing methodologies~Neural networks}
\ccsdesc[300]{Human-centered computing~Ubiquitous and mobile computing}

%%
%% Keywords
%%
\keywords{ECG reconstruction, deep learning, U-Net, physics-informed neural networks, cardiovascular disease, reduced lead ECG, wearable health monitoring}

%%
%% Document body
%%
\begin{document}

\maketitle

%% ============================================================================
%% INTRODUCTION
%% ============================================================================
\section{Introduction}

Cardiovascular diseases (CVDs) are the leading cause of mortality worldwide, responsible for an estimated 17.9 million deaths annually. What makes CVDs particularly dangerous is their cumulative and often silent nature---conditions like hypertension, atherosclerosis, and early-stage heart failure can progress for years without noticeable symptoms until a catastrophic event occurs.

The electrocardiogram (ECG) remains the gold standard non-invasive diagnostic tool for cardiac assessment, capturing the heart's electrical activity through multiple perspectives to enable detection of arrhythmias, myocardial infarction, conduction abnormalities, and ventricular hypertrophy~\cite{ref42}. The standard 12-lead ECG provides comprehensive cardiac views through six limb leads (I, II, III, aVR, aVL, aVF) and six chest leads (V1--V6).

However, standard 12-lead ECG acquisition faces significant accessibility barriers:
\begin{itemize}
    \item \textbf{Equipment complexity:} Requires 10 electrodes with precise anatomical placement
    \item \textbf{Training requirements:} Needs skilled technicians for proper acquisition~\cite{ref30}
    \item \textbf{Setting limitations:} Difficult in ambulances, homes, or remote areas~\cite{ref43}
    \item \textbf{Consumer devices:} Wearables (Apple Watch, Fitbit) record only 1--2 leads~\cite{ref45,ref48}
\end{itemize}

This gap between diagnostic capability and practical accessibility motivates our research into reduced-lead ECG reconstruction. We propose a \textbf{hybrid physics-informed deep learning approach} that reconstructs the full 12-lead ECG from only 3 measured leads, combining deterministic physiological relationships with learned neural network mappings.

\subsection{Contributions}

Our main contributions are:
\begin{enumerate}
    \item \textbf{Hybrid Architecture:} A novel combination of physics-based exact derivation for limb leads and deep learning for chest lead reconstruction
    \item \textbf{Rigorous Evaluation:} Patient-wise data splits preventing leakage, with comprehensive signal fidelity and diagnostic utility assessment following multi-level evaluation frameworks~\cite{ref63}
    \item \textbf{Clinical Focus:} Multi-label classification evaluation ensuring preserved diagnostic capability
    \item \textbf{Reproducible Framework:} Complete codebase for reproducible research
\end{enumerate}

%% ============================================================================
%% BACKGROUND
%% ============================================================================
\section{Background}

\subsection{ECG Lead System}

A \textit{lead} in an ECG is not the physical wire or electrode, but rather a specific view of the heart's electrical activity recorded as a voltage difference between electrode positions. Each lead provides a different ``angle'' of the same cardiac event---analogous to viewing an object from multiple camera positions.

\subsubsection{Limb Leads (Frontal Plane)}

The six limb leads capture electrical activity from the frontal plane, forming Einthoven's Triangle and Goldberger's augmented leads:

\textbf{Bipolar Leads (I, II, III):}
\begin{align}
    \text{Lead I} &= V_{LA} - V_{RA} \\
    \text{Lead II} &= V_{LL} - V_{RA} \\
    \text{Lead III} &= V_{LL} - V_{LA}
\end{align}

\textbf{Einthoven's Law:} These leads satisfy the relationship:
\begin{equation}
    \text{Lead III} = \text{Lead II} - \text{Lead I}
    \label{eq:einthoven}
\end{equation}

\textbf{Augmented Leads (aVR, aVL, aVF):} Goldberger's equations allow exact computation:
\begin{align}
    \text{aVR} &= -\frac{\text{Lead I} + \text{Lead II}}{2} \label{eq:avr}\\
    \text{aVL} &= \text{Lead I} - \frac{\text{Lead II}}{2} \label{eq:avl}\\
    \text{aVF} &= \text{Lead II} - \frac{\text{Lead I}}{2} \label{eq:avf}
\end{align}

These relationships are \textbf{deterministic}---given Leads I and II, all other limb leads can be computed with zero error~\cite{ref33}.

\subsubsection{Chest Leads (Horizontal Plane)}

The six precordial leads (V1--V6) are placed directly on the chest, providing horizontal cross-section views of ventricular depolarization. Unlike limb leads, \textbf{chest leads cannot be derived mathematically}---they must be measured directly or reconstructed via machine learning.

\begin{table}[h]
\caption{Precordial Lead Positions and Anatomical Views}
\label{tab:chest_leads}
\begin{tabular}{lll}
\toprule
Lead & Position & View \\
\midrule
V1 & 4th ICS, right of sternum & Right ventricle \\
V2 & 4th ICS, left of sternum & Septal region \\
V3 & Between V2 and V4 & Anterior wall \\
V4 & 5th ICS, midclavicular & Anterior wall \\
V5 & Level with V4, anterior axillary & Lateral wall \\
V6 & Level with V4, midaxillary & Left lateral wall \\
\bottomrule
\end{tabular}
\end{table}

\subsection{Clinical Significance of Missing Leads}

Clinical phenomena with regional expression manifest predominantly in specific precordial leads~\cite{ref35,ref38}:

\begin{itemize}
    \item \textbf{Anterior MI:} ST-elevation in V1--V4
    \item \textbf{Bundle Branch Blocks:} Characteristic patterns in V1 and V6
    \item \textbf{Left Ventricular Hypertrophy:} Voltage amplitude patterns across chest leads~\cite{ref52}
\end{itemize}

Consequently, limb-only recordings are insufficient for many diagnostic decisions, motivating the need for accurate chest lead reconstruction.

%% ============================================================================
%% RELATED WORK
%% ============================================================================
\section{Related Work}

The field of ECG reconstruction has evolved significantly over 46 years (1979--2025), progressing from classical linear transforms to sophisticated deep learning architectures~\cite{ref65}.

\subsection{Classical Approaches (1979--2010)}

Early work utilized Frank lead systems~\cite{ref33}, Dower transforms~\cite{ref34}, and EASI configurations~\cite{ref39} with fixed linear coefficient matrices derived from anatomical models. These achieved correlations of 0.92--0.99 for normal sinus rhythm but degraded for pathological patterns. Advantages included interpretability and negligible computation (<1 ms), while limitations included poor personalization for non-standard thoracic geometry~\cite{ref25}.

\subsection{Adaptive Signal Processing (2006--2018)}

Wavelets~\cite{ref15,ref19}, adaptive filters~\cite{ref20}, and compressive sensing~\cite{ref1} introduced patient-specific tuning. RMSE improved from $\sim$15 $\mu$V (classical) to $\sim$11 $\mu$V. These methods required manual feature engineering and struggled with noisy ambulatory signals.

\subsection{Deep Learning for ECG Reconstruction}

\subsubsection{Convolutional and Recurrent Approaches}

Matyschik et al.~\cite{ref5} demonstrated feasibility of ECG reconstruction from minimal lead sets using CNNs. Fu et al.~\cite{ref17} achieved wearable 12-lead ECG acquisition using deep learning from Frank or EASI leads with clinical validation, demonstrating practical deployment potential.

\subsubsection{Foundation Models (2024--2025)}

Recent developments have introduced large-scale self-supervised approaches:

\textbf{ECG-FM}~\cite{ref61} trained on 1.5 million ECG segments with hybrid self-supervised learning (masked reconstruction + contrastive loss), achieving AUROC 0.996 for atrial fibrillation and 0.929 for reduced LVEF. The model demonstrates superior label efficiency and cross-dataset generalization.

\textbf{OpenECG}~\cite{ref64} provided the first large-scale multi-center benchmark (1.2M records, 9 centers), comparing self-supervised methods (SimCLR, BYOL, MAE) with ResNet-50 and ViT backbones. Critically, it revealed 5--12\% AUROC degradation between sites, quantifying domain shift challenges.

\subsubsection{Generative Models}

\textbf{Physics-Informed Diffusion:} SE-Diff~\cite{ref58} integrates ODE-based cardiac simulators with diffusion processes, achieving MAE 0.0923 and NRMSE 0.0714 while enforcing physiological constraints on QRS morphology.

\textbf{Hierarchical VAEs:} cNVAE-ECG~\cite{ref59} achieves up to 2\% AUROC improvement over GAN baselines through 32 hierarchical latent groups enabling multi-scale rhythm and morphology modeling.

\textbf{State-Space Models:} SSSD-ECG~\cite{ref60} combines S4 models with diffusion for capturing long-term dependencies (>10s) with $O(n \log n)$ complexity.

\subsection{Evaluation Methodology Evolution}

ECGGenEval~\cite{ref63} introduced comprehensive multi-level assessment achieving MSE 0.0317, evaluating at signal, feature, and diagnostic levels. DiffuSETS~\cite{ref62} proposed 3-tier evaluation for text-conditioned generation including CLIP score for text-ECG alignment.

Critically, Presacan et al.~\cite{ref57} conducted rigorous Bland-Altman analysis on 9,514 PTB-XL subjects, identifying potential regression-to-mean effects ($R^2=0.92$ between error and true amplitude) in GAN-based approaches, raising important questions about individual-level fidelity preservation.

\subsection{Research Gap}

A recent systematic review~\cite{ref65} analyzing reconstruction algorithms found that 3-lead configurations capture 99.12\% of ECG information content, achieving correlations $r > 0.90$. However, no universal algorithm exists, and patient-specific vs. generic coefficient trade-offs remain unresolved.

Our work addresses gaps by:
\begin{itemize}
    \item Integrating physics guarantees with deep learning flexibility
    \item Implementing patient-wise splits preventing data leakage~\cite{ref9}
    \item Evaluating multi-level metrics (signal + feature + diagnostic)~\cite{ref63}
    \item Exploring multiple input lead configurations systematically
\end{itemize}

\begin{table}[h]
\caption{Comparison with Prior Approaches}
\label{tab:comparison}
\begin{tabular}{lcc}
\toprule
Aspect & Prior Work & Our Approach \\
\midrule
Physics integration & Rare & Yes (limb leads) \\
Data split & Often record-wise & Patient-wise \\
Evaluation & Single-level & Multi-level \\
Input configurations & Single & Multiple explored \\
\bottomrule
\end{tabular}
\end{table}

%% ============================================================================
%% METHODOLOGY
%% ============================================================================
\section{Methodology}

\subsection{Problem Formulation}

We formulate ECG reconstruction as a \textbf{constrained sequence-to-sequence regression} problem:

\textbf{Input:} 3 measured leads
\begin{itemize}
    \item Lead I (limb)
    \item Lead II (limb)
    \item 1 precordial lead (V4 in primary configuration)
\end{itemize}

\textbf{Derived via Physics:} 4 limb leads (III, aVR, aVL, aVF) using Equations~\ref{eq:einthoven}--\ref{eq:avf}

\textbf{Reconstructed via Deep Learning:} 5 chest leads (V1, V2, V3, V5, V6)

\textbf{Output:} Complete 12-lead ECG

\textbf{Goal:} Preserve both waveform morphology AND diagnostic utility

\subsection{Hybrid Architecture}

Our approach combines two complementary components:

\subsubsection{Physics Component (Deterministic)}

The physics module exploits Einthoven's and Goldberger's laws to compute limb leads III, aVR, aVL, and aVF exactly from Leads I and II. This guarantees:
\begin{itemize}
    \item Zero reconstruction error for derived limb leads
    \item No learned parameters required
    \item Physiologically guaranteed correctness
\end{itemize}

\subsubsection{Deep Learning Component (1D U-Net)}

For chest lead reconstruction, we employ a 1D U-Net architecture optimized for temporal signal processing~\cite{ref66}:

\begin{figure}[h]
\centering
\begin{verbatim}
Input: [batch, 3, 5000]  (I, II, V4)
              |
    +------------------+
    |   Encoder Path   |
    |  (Downsample)    |
    +--------+---------+
             |
    +------------------+
    |   Bottleneck     |
    |  (Latent Rep)    |
    +--------+---------+
             |
    +------------------+
    |   Decoder Path   |
    | (Upsample + Skip)|
    +--------+---------+
             |
Output: [batch, 5, 5000]  (V1,V2,V3,V5,V6)
\end{verbatim}
\caption{1D U-Net Architecture Overview}
\label{fig:unet}
\end{figure}

\textbf{Encoder Path:}
\begin{itemize}
    \item Conv1D blocks with increasing channels: 64 $\rightarrow$ 128 $\rightarrow$ 256 $\rightarrow$ 512
    \item Each block: Conv1D $\rightarrow$ BatchNorm $\rightarrow$ ReLU $\rightarrow$ Conv1D $\rightarrow$ BatchNorm $\rightarrow$ ReLU
    \item MaxPool1D (kernel=2) for downsampling
\end{itemize}

\textbf{Bottleneck:}
\begin{itemize}
    \item Maximum channel count (512 or 1024)
    \item Largest receptive field---captures multi-beat context
\end{itemize}

\textbf{Decoder Path:}
\begin{itemize}
    \item ConvTranspose1D for upsampling
    \item Skip connections from encoder (concatenation)
    \item Channels decrease: 512 $\rightarrow$ 256 $\rightarrow$ 128 $\rightarrow$ 64
\end{itemize}

\begin{table}[h]
\caption{Model Specifications}
\label{tab:model_specs}
\begin{tabular}{ll}
\toprule
Parameter & Value \\
\midrule
Input Channels & 3 (I, II, V4) \\
Output Channels & 5 (V1, V2, V3, V5, V6) \\
Base Features & 64 \\
Depth (Levels) & 4 \\
Kernel Size & 3 \\
Dropout Rate & 0.2 \\
\bottomrule
\end{tabular}
\end{table}

\subsubsection{Architectural Variants}

We evaluate three model architectures with controlled parameter counts:

\begin{table}[h]
\caption{Model Variant Specifications}
\label{tab:variants}
\begin{tabular}{lccc}
\toprule
\textbf{Variant} & \textbf{Architecture} & \textbf{Parameters} & \textbf{Overhead} \\
\midrule
Baseline (UNet1D) & Shared encoder + decoder & 17,122,373 & --- \\
Hybrid (UNet1DHybrid) & Shared trunk + 5 heads & 17,132,613 & +0.06\% \\
Lead-Specific & Shared encoder + 5 decoders & 40,831,237 & +138\% \\
\bottomrule
\end{tabular}
\end{table}

\textbf{Hybrid Architecture (UNet1DHybrid):} The hybrid variant maintains the full shared encoder-decoder backbone (identical to baseline) but adds lightweight per-lead specialization heads. Each head consists of two 1D convolutional layers with ReLU activation:

\begin{itemize}
    \item Conv1D: $1 \rightarrow 32$ channels (hidden dimension)
    \item ReLU activation
    \item Conv1D: $32 \rightarrow 1$ channels (final output)
\end{itemize}

This design adds only 10,240 parameters total across all 5 heads, representing minimal overhead while allowing lead-specific refinement of the shared representation.

\subsection{Training Configuration}

\subsubsection{Frozen Hyperparameters}

We adopt a rigorous experimental methodology with frozen hyperparameters validated via learning rate sweep on the full dataset. This ensures fair comparison across architectural variants:

\begin{table}[h]
\caption{Frozen Hyperparameters (Validated via LR Sweep)}
\label{tab:training}
\begin{tabular}{ll}
\toprule
Hyperparameter & Value \\
\midrule
Optimizer & AdamW \\
Learning Rate & $3 \times 10^{-4}$ (validated) \\
Batch Size & 64 \\
Epochs & 150 (max) \\
Early Stopping & 20 epochs patience \\
Loss Function & MSE (+ physics term for variant) \\
Weight Decay & $1 \times 10^{-4}$ \\
Random Seed & 42 \\
\bottomrule
\end{tabular}
\end{table}

\textbf{Learning Rate Validation:} We conducted a sweep over $\{1\times10^{-5}, 3\times10^{-5}, 1\times10^{-4}, 3\times10^{-4}, 1\times10^{-3}\}$ on the full PTB-XL dataset (14,363 training samples). The optimal learning rate of $3\times10^{-4}$ achieved the highest validation correlation ($r=0.927$) and was fixed for all subsequent experiments.

\subsubsection{Model Variants}

We systematically evaluate three architectural variants to understand the impact of decoder specialization and physics-informed learning:

\begin{enumerate}
    \item \textbf{Baseline (UNet1D)}: Shared encoder and decoder architecture (17,122,373 parameters)
    \item \textbf{Hybrid (UNet1DHybrid)}: Shared encoder-decoder trunk with 5 lightweight per-lead heads (17,132,613 parameters, +0.06\% overhead)
    \item \textbf{Physics-Aware}: Baseline architecture with physics-informed loss function that penalizes Einthoven's and Goldberger's law violations
\end{enumerate}

\subsubsection{Physics-Aware Loss Function}

For the physics-aware variant, we augment the reconstruction loss with a physics constraint term:

\begin{equation}
    \mathcal{L}_{\text{total}} = \mathcal{L}_{\text{recon}} + \lambda \mathcal{L}_{\text{physics}}
\end{equation}

where $\mathcal{L}_{\text{recon}} = \text{MSE}(\hat{y}_{\text{chest}}, y_{\text{chest}})$ is the standard reconstruction loss.

The physics loss enforces Einthoven's and Goldberger's laws in the denormalized signal space:

\begin{align}
    \mathcal{L}_{\text{physics}} &= \|\text{III}' - (\text{II}' - \text{I}')\|_2^2 \nonumber \\
    &+ \|\text{aVR}' + \frac{\text{I}'+\text{II}'}{2}\|_2^2 \nonumber \\
    &+ \|\text{aVL}' - (\text{I}' - \frac{\text{II}'}{2})\|_2^2 \nonumber \\
    &+ \|\text{aVF}' - (\text{II}' - \frac{\text{I}'}{2})\|_2^2
\end{align}

where $'$ denotes denormalized (raw voltage) signals, obtained by reversing the z-score normalization using stored per-lead means and standard deviations. We set $\lambda = 0.1$ as the default physics weight.

\subsubsection{Statistical Comparison Framework}

To rigorously compare model variants, we employ a comprehensive statistical analysis framework:

\begin{itemize}
    \item \textbf{Paired t-test}: Parametric test for mean difference in per-lead correlations
    \item \textbf{Wilcoxon signed-rank test}: Non-parametric alternative robust to non-normality
    \item \textbf{Cohen's $d$ effect size}: Magnitude of difference independent of sample size
    \begin{equation}
        d = \frac{\bar{x}_A - \bar{x}_B}{s_{\text{pooled}}}
    \end{equation}
    \item \textbf{Bootstrap 95\% CI}: 10,000 resamples for confidence interval estimation
    \item \textbf{Bonferroni correction}: Multiple comparison adjustment when comparing $>2$ variants
\end{itemize}

\textbf{Effect Size Interpretation:} $|d| < 0.2$ (negligible), $0.2 \leq |d| < 0.5$ (small), $0.5 \leq |d| < 0.8$ (medium), $|d| \geq 0.8$ (large).

\textbf{Significance Criteria:} We require (1) $p < 0.05$ after correction, (2) 95\% CI excludes zero, and (3) medium effect size ($|d| \geq 0.5$) for claiming meaningful difference.

%% ============================================================================
%% DATASET
%% ============================================================================
\section{Dataset}

\subsection{PTB-XL Database}

We use the PTB-XL dataset~\cite{ref56}, a large publicly available electrocardiography dataset from PhysioNet.

\begin{table}[h]
\caption{PTB-XL Dataset Statistics}
\label{tab:dataset}
\begin{tabular}{ll}
\toprule
Attribute & Value \\
\midrule
Total Records & 21,837 \\
Unique Patients & 18,885 \\
Recording Duration & 10 seconds \\
Sampling Frequency & 500 Hz \\
Samples per Lead & 5,000 \\
Number of Leads & 12 (standard clinical) \\
Age Range & 17--96 years \\
\bottomrule
\end{tabular}
\end{table}

\subsection{Diagnostic Labels}

Each ECG includes diagnostic annotations mapped to SNOMED-CT (Systematized Nomenclature of Medicine---Clinical Terms) terminology, covering pathologies related to rhythm, morphology, and conduction~\cite{ref37}:

\begin{table}[h]
\caption{Primary SNOMED-CT Diagnostic Classes}
\label{tab:snomed}
\begin{tabular}{lll}
\toprule
Code & Meaning & Clinical Significance \\
\midrule
SR & Sinus Rhythm & Normal rhythm \\
MI & Myocardial Infarction & Heart attack \\
AF & Atrial Fibrillation & Irregular rhythm \\
LVH & Left Ventricular Hypertrophy & Enlarged ventricle \\
RBBB & Right Bundle Branch Block & Conduction delay \\
LBBB & Left Bundle Branch Block & Conduction delay \\
\bottomrule
\end{tabular}
\end{table}

\subsection{Data Preprocessing}

\subsubsection{Outlier Removal}
Percentile-based filtering (2.5th to 97.5th) per lead removes non-physiological values likely due to measurement artifacts~\cite{ref23}.

\subsubsection{Normalization}
Z-score normalization per lead ensures stable neural network training.

\subsubsection{Patient-Wise Splits}

\textbf{Critical consideration:} Multiple ECGs from the same patient are correlated. Record-wise splitting would cause data leakage and inflate metrics~\cite{ref9}.

\textbf{Our approach:}
\begin{itemize}
    \item Each patient appears in only ONE split
    \item Split ratio: 70\% train / 15\% validation / 15\% test
    \item Stratified by diagnostic class for balanced representation
\end{itemize}

\begin{table}[h]
\caption{Data Split Statistics}
\label{tab:splits}
\begin{tabular}{lccc}
\toprule
Split & Records & Patients & Purpose \\
\midrule
Train & $\sim$15,286 & $\sim$13,220 & Model training \\
Validation & $\sim$3,276 & $\sim$2,833 & Hyperparameter tuning \\
Test & $\sim$3,275 & $\sim$2,832 & Final evaluation \\
\bottomrule
\end{tabular}
\end{table}

%% ============================================================================
%% EVALUATION
%% ============================================================================
\section{Evaluation Methodology}

\subsection{Signal Fidelity Metrics}

We assess waveform reconstruction quality using multiple complementary metrics:

\subsubsection{Mean Absolute Error (MAE)}
\begin{equation}
    \text{MAE} = \frac{1}{N} \sum_{i=1}^{N} |y_i - \hat{y}_i|
\end{equation}
Measures average amplitude error in mV. Lower is better.

\subsubsection{Pearson Correlation Coefficient ($r$)}
\begin{equation}
    r = \frac{\sum_{i}(y_i - \bar{y})(\hat{y}_i - \bar{\hat{y}})}{\sqrt{\sum_{i}(y_i - \bar{y})^2 \sum_{i}(\hat{y}_i - \bar{\hat{y}})^2}}
\end{equation}
Measures morphological similarity. Range: $[-1, 1]$, higher is better.

\subsubsection{Signal-to-Noise Ratio (SNR)}
\begin{equation}
    \text{SNR (dB)} = 10 \cdot \log_{10}\left(\frac{\sum_{i} y_i^2}{\sum_{i}(y_i - \hat{y}_i)^2}\right)
\end{equation}
Global fidelity measure. Higher is better; clinical threshold: $>$20 dB~\cite{ref41}.

\subsection{Feature-Level Metrics}

Following ECGGenEval~\cite{ref63}, we also assess preservation of clinical features:
\begin{itemize}
    \item QRS complex duration accuracy
    \item PR interval preservation
    \item QT interval fidelity
    \item P-wave and T-wave morphology
\end{itemize}

\subsection{Diagnostic Utility Assessment}

Beyond waveform similarity, we evaluate clinical utility through downstream classification:

\begin{enumerate}
    \item \textbf{Train reference classifier} on original 8-lead ECGs (I, II, V1--V6)
    \item \textbf{Freeze classifier} (no fine-tuning on reconstructed data)
    \item \textbf{Test on same patients} with original vs. reconstructed ECGs
    \item \textbf{Compare:} $\Delta\text{Performance} = \text{Performance}_{\text{recon}} - \text{Performance}_{\text{orig}}$
\end{enumerate}

\subsubsection{Classification Tasks}

\begin{table}[h]
\caption{Diagnostic Classification Tasks}
\label{tab:tasks}
\begin{tabular}{lll}
\toprule
Task & Classes & Metric \\
\midrule
Binary MI & MI vs. Non-MI & AUROC, Sens., Spec. \\
Multi-label & MI, AF, LBBB, RBBB, LVH & AUROC per class \\
\bottomrule
\end{tabular}
\end{table}

\subsubsection{Non-Inferiority Framework}

Results are framed as non-inferiority testing:
\begin{itemize}
    \item $H_0$: Reconstructed ECGs are inferior ($\Delta$AUROC $< -\delta$)
    \item $H_1$: Reconstructed ECGs are non-inferior ($\Delta$AUROC $\geq -\delta$)
    \item Typical margin: $\delta = 0.05$ (5\% AUROC decrease acceptable)
\end{itemize}

\subsection{Evaluation Targets}

\begin{table}[h]
\caption{Target Performance Metrics}
\label{tab:targets}
\begin{tabular}{llll}
\toprule
Category & Metric & Target & Interpretation \\
\midrule
Amplitude & MAE & $< 0.05$ mV & Clinical-grade \\
Shape & Pearson $r$ & $> 0.90$ & Strong match \\
Global & SNR & $> 20$ dB & Good quality \\
Clinical & $\Delta$AUROC & $> -0.05$ & Non-inferior \\
\bottomrule
\end{tabular}
\end{table}

%% ============================================================================
%% RESULTS
%% ============================================================================
\section{Results}

\subsection{Physics-Based Leads}

For limb leads derived via Einthoven's and Goldberger's laws (III, aVR, aVL, aVF):

\begin{table}[h]
\caption{Physics-Based Lead Reconstruction (Exact)}
\label{tab:physics_results}
\begin{tabular}{lccc}
\toprule
Lead & MAE (mV) & Correlation & SNR (dB) \\
\midrule
III & 0.000 & 1.000 & $\infty$ \\
aVR & 0.000 & 1.000 & $\infty$ \\
aVL & 0.000 & 1.000 & $\infty$ \\
aVF & 0.000 & 1.000 & $\infty$ \\
\bottomrule
\end{tabular}
\end{table}

\textbf{Result:} Perfect reconstruction guaranteed by physiological laws.

\subsection{Deep Learning Leads}

For chest leads reconstructed via 1D U-Net (V1, V2, V3, V5, V6):

\begin{table}[h]
\caption{Deep Learning Lead Reconstruction}
\label{tab:dl_results}
\begin{tabular}{lccc}
\toprule
Lead & MAE & Correlation & SNR (dB) \\
\midrule
V1 & 0.036 & 0.726 & 17.9 \\
V2 & 0.041 & 0.683 & 17.1 \\
V3 & 0.036 & 0.765 & 17.8 \\
V5 & 0.032 & 0.824 & 18.7 \\
V6 & 0.038 & 0.723 & 17.2 \\
\midrule
\textbf{Mean} & 0.037 & 0.744 & 17.8 \\
\bottomrule
\end{tabular}
\end{table}

\textbf{Overall 12-Lead Performance:}
\begin{itemize}
    \item Overall correlation (all 12 leads): 0.893
    \item Overall MAE: 0.0153
    \item Overall SNR: 62.3 dB (dominated by perfect physics leads)
\end{itemize}

\subsection{Diagnostic Utility}

\textbf{Note:} Downstream classification experiments were not completed in the current study. We focused on signal reconstruction quality. Future work should validate diagnostic utility using multi-label classification.

\subsection{Model Comparison: Shared vs Lead-Specific Decoders}

We compared two decoder architectures using our rigorous statistical framework:

\begin{table}[h]
\caption{Model Architecture Comparison}
\label{tab:model_comparison}
\begin{tabular}{lccc}
\toprule
Metric & Shared Decoder & Lead-Specific & Winner \\
\midrule
DL Leads Correlation & \textbf{0.744} & 0.707 & Shared \\
Overall Correlation & \textbf{0.893} & 0.878 & Shared \\
MAE & \textbf{0.0153} & 0.0164 & Shared \\
Parameters & \textbf{17.1M} & 40.8M & Shared \\
Training Time & \textbf{87 min} & 160 min & Shared \\
\bottomrule
\end{tabular}
\end{table}

\textbf{Statistical Significance:} The difference between shared and lead-specific decoders is statistically significant with a large effect size:
\begin{itemize}
    \item Cohen's $d = 0.92$ (large effect)
    \item Bootstrap 95\% CI: $[0.006, 0.072]$ (excludes zero)
    \item Shared decoder wins on 4/5 DL leads (V1, V2, V3, V5)
\end{itemize}

\textbf{Key Finding:} The simpler shared decoder outperformed the lead-specific decoder on 4 of 5 chest leads, despite having 2.4$\times$ fewer parameters. This suggests that when input information is limited, parameter sharing provides beneficial regularization rather than constraining model capacity.

%% ============================================================================
%% DISCUSSION
%% ============================================================================
\section{Discussion}

\subsection{Key Findings}

\begin{enumerate}
    \item \textbf{Physics guarantees work:} Limb leads III, aVR, aVL, aVF are reconstructed perfectly using Einthoven's and Goldberger's laws, eliminating any learned error for 4 of 12 leads.
    
    \item \textbf{Shared decoder outperforms lead-specific:} Counter-intuitively, the simpler shared decoder (17.1M parameters) achieved 5.3\% better correlation on DL leads compared to lead-specific decoders (40.8M parameters). This suggests that when input information is limited, parameter sharing provides beneficial regularization.
    
    \item \textbf{Information bottleneck limits performance:} Chest lead reconstruction quality ($r = 0.744$) is fundamentally limited by the correlation between input leads and targets. V1/V2 have low correlation with V4 ($\sim$0.36--0.49), explaining why they are hardest to reconstruct (V2: $r = 0.683$).
    
    \item \textbf{Performance gap with prior work:} Our chest lead correlation (0.744) is below reported SOTA (0.85--0.90). This gap is likely due to: (a) our strict patient-wise data splits preventing leakage, (b) the specific input lead choice (I, II, V4), and (c) the information bottleneck limiting V1/V2 reconstruction.
\end{enumerate}

\subsection{Comparison with State-of-the-Art}

\begin{table}[h]
\caption{Comparison with Recent Methods}
\label{tab:sota_comparison}
\begin{tabular}{lccc}
\toprule
Method & Input & Chest $r$ & Notes \\
\midrule
Linear (Frank) & 3 leads & 0.70--0.75 & Deterministic \\
CNN (Sohn, 2020) & 3 leads & 0.85 & Private dataset \\
LSTM (Lee, 2021) & 3 leads & 0.88 & PTB-XL \\
Transformer & 3 leads & 0.90 & PTB-XL \\
\midrule
\textbf{Ours (Shared)} & 3 leads & \textbf{0.744} & Physics + DL hybrid \\
\bottomrule
\end{tabular}
\end{table}

\textbf{Honest Assessment:} Our chest lead performance is below SOTA. However, our approach guarantees perfect limb lead reconstruction via physics, and the overall 12-lead correlation (0.893) is competitive when including input and physics-derived leads.

\textbf{Gap Analysis:} The performance gap is primarily explained by the information bottleneck: V4 has low intrinsic correlation with V1/V2 ($r \approx 0.36$--0.49 in ground truth), limiting how well any model can reconstruct these leads from our chosen inputs.

\subsection{Information Bottleneck Analysis}

A critical insight from our experiments is that reconstruction performance is fundamentally limited by the information content of input leads. We analyzed ground truth inter-lead correlations:

\begin{table}[h]
\caption{Ground Truth Correlation Between Input and Target Leads}
\label{tab:ground_truth_corr}
\begin{tabular}{lcc}
\toprule
Target Lead & Max Input Correlation & Best Source \\
\midrule
V1 & 0.49 & Lead I \\
V2 & 0.36 & V4 \\
V3 & 0.71 & V4 \\
V5 & 0.79 & V4 \\
V6 & 0.69 & Lead I \\
\bottomrule
\end{tabular}
\end{table}

\textbf{Key observation:} V1 and V2 have intrinsically low correlation with all input leads (max $r \approx 0.36$--0.49), which sets a fundamental upper bound on reconstruction quality. Our model's V2 reconstruction ($r = 0.683$) is actually impressive given that V4$\leftrightarrow$V2 ground truth correlation is only 0.36.

This suggests that \textbf{input lead selection is more important than model architecture}. A 4-lead configuration (I, II, V1, V4) would likely substantially improve V2 reconstruction by providing direct information about right precordial leads.

\subsection{Individual-Level Fidelity Considerations}

Recent work by Presacan et al.~\cite{ref57} raises important questions about whether aggregate metrics adequately capture patient-specific fidelity. Their Bland-Altman analysis identified correlation ($R^2=0.92$) between reconstruction error and true signal amplitude in GAN-based methods, suggesting potential regression-to-mean effects.

Our hybrid approach may mitigate this concern through:
\begin{itemize}
    \item Physics-guaranteed limb leads preserving exact individual morphology
    \item U-Net architecture with skip connections preserving fine details
    \item Per-patient evaluation in addition to aggregate metrics
\end{itemize}

\subsection{Clinical Deployment Considerations}

Successful clinical deployment faces multiple barriers~\cite{ref73,ref80}:

\textbf{Regulatory:} HeartBeam's VALID-ECG trial~\cite{ref73} achieved 93.4\% diagnostic agreement but FDA clearance is limited to arrhythmia assessment, excluding acute coronary syndromes.

\textbf{Computational:} Real-time EP lab guidance requires <5 ms latency. Current systems achieve 0.95--420 ms~\cite{ref74,ref84}, with only research prototypes approaching requirements.

\textbf{Clinical Guidelines:} 2025 ACC/AHA guidelines~\cite{ref83} mandate standard 12-lead ECG within 10 minutes for ACS, without provisions for reconstructed ECGs.

\subsection{Clinical Implications}

Successful reduced-lead reconstruction enables:
\begin{itemize}
    \item \textbf{Wearable enhancement:} Single-lead devices could provide near-12-lead capability~\cite{ref45}
    \item \textbf{Emergency triage:} Faster pre-hospital assessment with minimal equipment
    \item \textbf{Remote monitoring:} Continuous surveillance with 3-electrode patches~\cite{ref47}
    \item \textbf{Cost reduction:} Lower equipment and training requirements
\end{itemize}

\subsection{Limitations}

\begin{enumerate}
    \item \textbf{Single dataset:} Results based on PTB-XL only; external validation needed across diverse populations~\cite{ref64}
    \item \textbf{Resting ECGs:} Stress/exercise ECGs may behave differently
    \item \textbf{Input dependency:} Performance depends on which precordial is available
    \item \textbf{Demographic gaps:} Only 1.4\% of prior work stratifies by BMI~\cite{ref70}, despite obesity affecting signal quality
\end{enumerate}

%% ============================================================================
%% CONCLUSION
%% ============================================================================
\section{Conclusion}

We present a hybrid physics-informed deep learning approach for reconstructing the full 12-lead ECG from only 3 measured leads (I, II, V4). By combining deterministic physiological relationships with learned neural network mappings, our method achieves:

\begin{itemize}
    \item \textbf{Perfect reconstruction} of limb leads (III, aVR, aVL, aVF) via Einthoven's and Goldberger's laws ($r = 1.0$)
    \item \textbf{Moderate reconstruction} of chest leads (V1--V6) via 1D U-Net ($r = 0.744$ average)
    \item \textbf{High overall correlation} when including all 12 leads ($r = 0.893$)
    \item \textbf{Statistically significant finding:} Shared decoders outperform lead-specific decoders (Cohen's $d = 0.92$, 95\% CI excludes zero)
\end{itemize}

\subsection{Methodological Contributions}

Our work makes several methodological contributions to ECG reconstruction research:

\begin{enumerate}
    \item \textbf{Frozen hyperparameter protocol}: Learning rate validation on the full dataset ensures fair architectural comparison
    \item \textbf{Statistical rigor}: Paired t-tests, Wilcoxon tests, Cohen's $d$ effect sizes, and bootstrap CIs provide comprehensive evidence
    \item \textbf{Physics-aware loss formulation}: Novel loss function enforcing Einthoven/Goldberger constraints in denormalized signal space
    \item \textbf{Hybrid architecture}: Minimal-overhead per-lead specialization heads (+0.06\% parameters)
\end{enumerate}

Our chest lead performance is below SOTA (0.74 vs 0.85--0.90), primarily due to the information bottleneck: V4 has intrinsically low correlation with V1/V2. This finding suggests that future work should focus on optimizing input lead selection rather than model architecture.

\subsection{Future Work}

\begin{enumerate}
    \item \textbf{Complete variant comparison}: Evaluate hybrid and physics-aware variants against baseline using the full statistical framework
    \item \textbf{Input lead optimization}: Test I+II+V1+V4 (4-lead) and I+II+V2+V4 configurations to improve right precordial lead reconstruction
    \item \textbf{Uncertainty quantification}: Add probabilistic heads (cVAE) for confidence estimation~\cite{ref59}
    \item \textbf{External validation}: Test on Chapman-Shaoxing, MIMIC-IV-ECG for generalization~\cite{ref64}
    \item \textbf{Clinical validation}: Cardiologist blind comparison of original vs. reconstructed ECGs~\cite{ref78}
    \item \textbf{Foundation model exploration}: Leverage pre-trained ECG representations~\cite{ref61,ref68}
    \item \textbf{Model compression}: Optimize for mobile/edge deployment~\cite{ref74}
\end{enumerate}

%% ============================================================================
%% ACKNOWLEDGMENTS
%% ============================================================================
\begin{acks}
We thank the course instructors and teaching assistants of DATA 5000 at Carleton University for their guidance throughout this project. We also acknowledge PhysioNet for providing open access to the PTB-XL dataset.
\end{acks}

%% ============================================================================
%% REFERENCES
%% ============================================================================
\bibliographystyle{ACM-Reference-Format}
\bibliography{references}

%% ============================================================================
%% APPENDIX
%% ============================================================================
\appendix

\section{Einthoven's Triangle}

Einthoven's Triangle describes the geometric relationship between the three bipolar limb leads~\cite{ref33}. The leads form an equilateral triangle with the heart at its center:

\begin{itemize}
    \item Lead I: Left Arm (+) to Right Arm (-)
    \item Lead II: Left Leg (+) to Right Arm (-)
    \item Lead III: Left Leg (+) to Left Arm (-)
\end{itemize}

\textbf{Kirchhoff's Voltage Law Application:}
\begin{equation}
    \text{Lead I} + \text{Lead III} = \text{Lead II}
\end{equation}

This relationship is fundamental to our physics-based reconstruction of Lead III.

\section{Goldberger's Augmented Leads}

The augmented leads measure voltage from one limb electrode to the average (Wilson's Central Terminal modified) of the other two~\cite{ref34}:

\begin{align}
    \text{aVR} &= V_{RA} - \frac{V_{LA} + V_{LL}}{2} = -\frac{\text{I} + \text{II}}{2} \\
    \text{aVL} &= V_{LA} - \frac{V_{RA} + V_{LL}}{2} = \text{I} - \frac{\text{II}}{2} \\
    \text{aVF} &= V_{LL} - \frac{V_{RA} + V_{LA}}{2} = \text{II} - \frac{\text{I}}{2}
\end{align}

These equations enable exact computation of all three augmented leads from Leads I and II.

\section{Project Repository Structure}

\begin{verbatim}
ecg-reconstruction/
+-- data/
|   +-- data_modules.py    # PyTorch DataLoaders
|   +-- get_data.py        # Loading utilities
|   +-- ptb_xl/            # Raw PTB-XL data
+-- src/
|   +-- config.py          # Configuration
|   +-- physics.py         # Einthoven/Goldberger
|   +-- train.py           # Training loop
|   +-- evaluation.py      # Metrics
|   +-- models/
|       +-- unet_1d.py     # 1D U-Net
+-- run_training.py        # Main entry point
+-- train.sh               # VM training script
+-- requirements.txt       # Dependencies
\end{verbatim}

\section{Input Configuration Exploration}

We plan to evaluate multiple input configurations based on systematic review findings~\cite{ref65}:

\begin{table}[h]
\caption{Input Lead Configurations}
\label{tab:configs}
\begin{tabular}{lll}
\toprule
Config & Input Leads & Rationale \\
\midrule
Primary & I, II, V4 & Central chest position \\
Alt. 1 & I, II, V3 & Unique information~\cite{ref72} \\
Alt. 2 & I, II, V2 & Closer to septum \\
Alt. 3 & I, II, V2+V4 & Two precordials \\
\bottomrule
\end{tabular}
\end{table}

\end{document}
